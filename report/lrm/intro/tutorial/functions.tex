\section{Functions}
% example with functions, function definition

\subsection{Defining and Declaring Functions}

Functions can be called with a simple syntax. The goal is to make it easy to pass arguments and specify types on those arguments, as well as the return type. All functions are defined by starting with the \lstinline|fun| keyword, followed by an identifier including the name, before an optional list of parameters.

\lstinputlisting[caption=functions]{"intro/tutorial/code/functions.hak"}


\subsection{Parameters and Arguments}
% example with function arguments
% primitive types - by value
% composite types - by reference

All arguments given to a function for a function call are passed by value, unless the reference symbol \lstinline|&| is written just before the argument, as shown in the below example. This allows a person to manipulate a value that was passed in directly, rather than receiving a copy of it the argument.

\lstinputlisting[caption=arguments]{"intro/tutorial/code/arguments.hak"}
