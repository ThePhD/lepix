\section{Arithmetic Expressions}
% Self-explanatory
% Don't forget things like x += 2 meaning x = x+2

\subsection{Binary Arithmetic Operations}
\begin{itemize}[before=\itshape, label={}]
	\item addition-expression:
	\begin{itemize}[before=\itshape, label={}]
		\item \lstinline@expression + expression@
	\end{itemize}
	\item subtraction-expression:
	\begin{itemize}[before=\itshape, label={}]
		\item \lstinline@expression - expression@
	\end{itemize}
	\item division-expression:
	\begin{itemize}[before=\itshape, label={}]
		\item \lstinline@expression / expression@
	\end{itemize}
	\item multiplication-expression:
	\begin{itemize}[before=\itshape, label={}]
		\item \lstinline@expression * expression@
	\end{itemize}
	\item modulus-expression:
	\begin{itemize}[before=\itshape, label={}]
		\item \lstinline@expression % expression@
	\end{itemize}
\end{itemize}
\begin{enumerate}
	\item Symbolic expression to perform the commonly understood mathematical operations on two operands.
	\item All operations are left-associative.
\end{enumerate}

\subsection{Unary Arithmetic Operations}
\begin{itemize}[before=\itshape, label={}]
	\item unary-minus-expression:
	\begin{itemize}[before=\itshape, label={}]
		\item \lstinline@-expression@
	\end{itemize}
\end{itemize}
\begin{enumerate}
	\item Unary minus is typically interpreted as negation of the single operand.
	\item All operations are left-associative.
\end{enumerate}
