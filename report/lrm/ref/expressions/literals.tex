\section{Literals}
% literals such as integers / float definitions and the like

\subsection{Kinds of Literals}\label{ssec:Kinds of Literals}
There are many kinds of literals. They are:

\begin{itemize}[before=\itshape, label={}]
	\item literal:
	\begin{itemize}[before=\itshape, label={}]
		\item boolean-literal
		\item integer-literal 
		\item floating-literal 
		\item string-literal
	\end{itemize}
\end{itemize}

\subsection{Boolean Literals}
\begin{enumerate}
	\item A boolean literal are the keywords \lstinline|true| or \lstinline|false|.
\end{enumerate}

\subsection{Integer Literals}\label{ssec: Integer Literals}
\begin{enumerate}
	\item An integer literal is a valid sequence of digits with some optional alpha characters that change the interpretation of the supplied literal.
	\item A decimal integer literal uses digits `\lstinline|0|' through `\lstinline|9|' to define a base-10 number.
	\item A hexidecimal integer literal uses digits `\lstinline|0|' through `\lstinline|9|', `\lstinline|A|' through `\lstinline|F|' (case insensitive) to define a base-16 number. It must be prefixed by \lstinline|0x| or \lstinline|0x|.
	\item An octal integer literal uses digits `\lstinline|0|' and `\lstinline|7|' to define a base-8 number. It must be prefixed by \lstinline|0c| or \lstinline|0C|.
	\item A binary integer literal uses digits 0 and 1 to define a base-2 number. It must be prefixed by \lstinline|0b| (case sensitive).
	\item An $n$-digit integer literal uses the characters below to define a base-$n$ number. It must be prefixed by \lstinline|0n| or \lstinline|0N|. It must be suffixed by \lstinline|#n|, where $n$ is the desired base. The character set defined for these bases goes up to 63 characters, giving a maximum arbitrary base of 63. The characters which are: 
	\begin{itemize}
		\item[] \lstinline|0 - 9, A - Z, a - z, _|
	\end{itemize}
	\item Arbitrary bases for $n$-digit must be base-10 numbers.
	\item Groups of digits may be separated by a \lstinline|'| and do not change the integer literal at all.
\end{enumerate}

\subsection{Floating Literals}\label{ssec: Floating Literals}
\begin{enumerate}
	\item A floating literal has two primary forms, utilizing digits as defined in \ref{ssec: Integer Literals}.
	\item \label{enum-item:first-form} The first form must have a dot `\lstinline|.|' preceded by an integer literal and/or suffixed by an integer literal. It must have one or the other, and may not omit both the prefixing or suffixing integer literal.
	\item The second form follows \ref{enum-item:first-form}, but includes the exponent symbol \lstinline|e| and another integer literal describing that exponent. Both the exponent and integer literal must be present in this form, but if the exponent is included then the dot is not necessary and may be prefixed with only an integer literal or just an integer literal and a dot.
\end{enumerate}

\subsection{String Literals}\label{ssec: String Literals}
\begin{enumerate}
	\item A string literal is started with a single `\lstinline|'|' or double `\lstinline|"|' quotation mark and does not end until the next matching single `\lstinline|'|' or double `\lstinline|"|' quotation mark character, with respect to what the string was started with. This includes any and all spacing characters, including newline characters.
	\item Newline characters in a multi-line string will be included in the string as an ASCII Line Feed \lstinline|\n| character.
	\item A string literal must remove the leading space on each line that are equivalent to all other lines in the text, and any empty leading space at the start of the string.
	\item A string literal may retain the any leading space and common indentation by prefixing the opening single or double quotation mark with an `\lstinline|R|'.
\end{enumerate}
