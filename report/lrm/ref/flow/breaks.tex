\section{break and continue}
\label{sec:breaks}
% The break keyword
% for (...) {
% 	break;
% }
% for (...) {
% 	for (...) {
% 		break 2;
% 	}
% }
% break can jump out of multiple scopes, to avoid boolean flag nightmares or using
% Immediately Invoked Function Expressions / Closures (IIFE/C)
% scope for the variable "x" lasts until the closing bracket "}"
Break and continue statements are used to exit a loop immediately, before the specified condition has been reached.

\subsection{\texorpdfstring{\lstinline|break|}{break}}
\begin{enumerate}
	\item Break statements exit the block of a loop immediately.
\begin{lstlisting}[numbers=none]
while (...) {
	statements_above
	break; 
	statements_below
}
\end{lstlisting}
	\item In the example above, \lstinline|statements_above| would be executed only once. The \lstinline|statements_below| would never be executed.\footnote{Break statements are usually included inside of an if statement within the loop to immediately exit on a particular condition.}
\end{enumerate}

\subsection{\texorpdfstring{\lstinline|break N|}{break N}}
\begin{enumerate}
	\item Break statements can be used to exit nested loops by jumping out of multiple scopes by adding an integral constant after the \lstinline|break| keyword.
	\item The example below will allow the user to break out of both for loops with only one break statement.\footnote{This could be considered a structured version of goto for loops and should be used with the programmer's utmost discretion.}
\begin{lstlisting}[numbers=none]
for (...) {
	for (...) {
		statements_above
		if (condition)
			break 2;
		statements_below
	}
}
\end{lstlisting}
\end{enumerate} 


\subsection{\texorpdfstring{\lstinline|continue|}{continue}}
\begin{enumerate}
	\item Continue statements jump to the end of the loop body and begin the next iteration.
\begin{lstlisting}[numbers=none]
for (...) {
	statements_above
	if (expressions...) {
		continue;
	}
	statements_below
}
\end{lstlisting}
	\item When a continue statement is executed, statements below the \lstinline|continue| keyword are not executed, and the loop post-action and condition are immediately re-evaluated.
	\item In the example above, \lstinline|statements_above| would always be executed. The \lstinline|statements_below| would be executed on iterations where the if condition was false, since when the \lstinline|if| condition were true execution would jump back to the for loop's top.
\end{enumerate}
