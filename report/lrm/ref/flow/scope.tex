\section{Blocks and Scope}
% A block is defined by (explicit or implicit) curly braces {}
% Each block defines a new scope that inherits the variables and names of the block that surrounds it, recursively up to the global program scope 
% if blocks, for blocks, while blocks all can introduce variables associated with their curly braces {} before the curly braces in the designated sequence points
% an entire program is implicitly surrounded by {}, defining the 'global block', which makes for the 'global scope'
% blocks in general inherit the named definitions of scopes that surround them, but things defined in a block do not escape outside of that block
Braces  \lstinline|{| and \lstinline|}| are used to group statements in to blocks. Braces that surround the contents of a function are an example of grouping statements like this. Statements in the body of a \lstinline|for|, \lstinline|while|, \lstinline|if| or \lstinline|switch| statement are also  surrounded in braces, and therefore also contained in a block. Variables declared within a block exist only in that block. A semicolon is not required after the right brace. 

\subsection{Blocks}
\label{subsec:blocks}
% a block can be introduced by simply introducing curly braces into the program at any point in any definitions
\begin{enumerate}
	\item At any point in a program, braces can be used to create a block. For example,
\begin{lstlisting}[numbers=none]
var x: int = 2;
var y: int = 4;
var result: int;
{
	var z: int = 6
	result = x + y + z;
}
\end{lstlisting}
	\item In this trivial example, the statements on lines 5 and 6 live within their own block.
	\item Blocks do have access to named definitions of their surrounding scope, however variables define within a block exist only within that block.
\end{enumerate}

\subsection{Scope}
% Scopes encompass: 
% - any names and identifiers defined within that scope
% this includes function, variable, and type definitions
\begin{enumerate}
	\item Scopes are defined as the collection of identifiers and available within the current lexicographic block\footnote{This is usually between two curly braces \lstinline|\{\}|}.
	\item Every program is implicitly surrounded by braces, which define the \textbf{global block}.
\end{enumerate}

\subsection {Variable Scope}
% variables last within the curly braces that surround them, except where defined for function declaration scope and control flow scope below (reference sections directly)
% TODO: section reference stuff; need to figure that out
% variables are constructed when they are first encountered, and destructed at the end of the scope (e.g., the "}") in the reverse order that they were encountered in
\begin{enumerate}
	\item Variables are in scope only within their own block\footnote{E.g., between the brackets \lstinline|\{\}|}.
	\item In the example in Section \ref{subsec:blocks}, \lstinline|z| is declared within the braces.
	\item Variables declared within blocks last only within lifetime of that block.
	\item If we attempted to access \lstinline|z| outside of this block, this would cause an error to occur.
	\item If a variable with a particular identifier has been declared and the identifier is re-used within a nested block.
	\item The original definition of the identifier is \textbf{shadowed} and the new one is used until the end of the block.
	\item Variables are constructed, that is, stored in memory when they are first encountered in their scope, and destructed at the scope's end in the reverse order they were encountered in.
\end{enumerate}

\subsection{Function Scope}
\begin{enumerate}
	\item Function definitions define a new block, which each have their own scope.
	\item Function definitions have access to any variables within their surrounding scope, however anything defined in the function definition's block is not accessible in the surrounding blocks.
	\item Variables defined in a parameter list belong to the definition-scope of the function.
\end{enumerate}

\subsection{Control Flow Scope}
\begin{enumerate}
	\item Control flow also introduces a new block with its own scope.
	\item Variables initialized in any control flow statement, that is within the parenthesis before the block, belong to the control flow block and are not accessible in the surrounding block.
	\item In the statement \lstinline|for (var x = 0 to 5) {...}|, \texttt{x} only exists within that for loop and destructed after the loop ends.
\end{enumerate}
 
 
