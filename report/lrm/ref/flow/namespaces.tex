\section{Namespaces}
% namespaces are a group of program elements (see: elements section) all defined and accessed through a single name
% namespaces do not introduce new scope: they merely provide named organization
% this is the only exception to the curly brace {} rule, where each pair introduces a new scope

\begin{enumerate}
	\item Namespaces are essentially blocks that allow identifiers to be prefix with an arbitrary nesting of names. They are declared with the \lstinline|namespace| keyword, followed by several identifiers delimited by a dot `.' symbol.
	\item Accessing variables and functions inside of a namespace must have the name of the namespace prefixed before the name of the desired identifier.
	\item The namespaces \lstinline|lib| and \lstinline|compiler| is reserved for use by standard library implementations and the compiler.
	\item Namespaces are the only bracket-delimited lexical scope that do not dictate the lifetime of the variables associated with them. These variables are part of the \textbf{global scope}.
\end{enumerate}