\section{Data Types}
% integral, floating point numerics
% int - 32 bit, only one we have
% float - 32 bit, only one we have
% arrays, multidimensional arrays
% image is just an alias for a 2D array
% maybe pixel, if we're desperate

The types of the language are divided into two categories: primitive types and data types derived from those primitive types. The primitive types are the \lstinline|boolean| type, the integral type \lstinline|int|, and the  floating-point type \lstinline|float|. The derived types are \lstinline|struct|, \lstinline|Array|, and \lstinline|image| and \lstinline|pixel|, which are both special instances of arrays.
\subsection{Primitive Data Types}\label{ssec:Primitive Data Types} 
\begin{enumerate}
\item \begin{verbatim} int \end{verbatim}
By default, the  \lstinline|int|
data type is a 32-bit signed two's complement integer, which has a minimum value of $-2^{31}$ 
and a maximum value of $2^{32}$.
\item \begin{verbatim} float \end{verbatim}
The \lstinline|float| data type is a single precision 32-bit IEEE 754 floating point. 
\item \begin{verbatim} boolean \end{verbatim}
The \lstinline|boolean| data type has possible values true and false. 
\end{enumerate} 
\subsection{Derived Data Types}\label{ssec: Derived Data Types} 
Besides the primitive data types, the derived types include arrays, structs, images, and pixels. 
\begin{enumerate}
\item \begin{verbatim} array \end{verbatim}
An array is a container object that holds a fixed number of values of a single type. Multi-dimensional arrays are also supported. They need to have arrays of the same length at each level. 
\item \begin{verbatim} pixel \end{verbatim}
A \lstinline|pixel| data type is a wrapper for an array that will contain the representation for each pixel of an image. It will contain the rgb values, each as a separate int, and the gray value of a pixel.
\item \begin{verbatim} image \end{verbatim}
The \lstinline|image| data type is just an alias for a 2-dimensional array. The 2-d array will define the size of an image and contains a pixel as each of its data elements. 
\item \begin{verbatim} struct \end{verbatim}
A structure is a collection of one or more variables, possibly of different types, grouped together under a single name for convenient handling. Structures help to organize data because they permit a group of related variables to be treated as a unit instead of as separate entities. 
\end{enumerate}

