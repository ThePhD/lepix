\section{Statements}
% Various keywords and identifier rules
Statements are executed sequentially in all cases except when explicit constructs for parallelization are used. Statements do not return values.

\begin{grammar}
<statement> ::= <expression_statement>
	\alt <branch_statement>
	\alt <compound_statement>
	\alt <iteration_statement>
	\alt <return_statement>
\end{grammar}

\subsection{Expression Statements}
\begin{grammar}
<expression_statement> ::= < >
	\alt <expression> ;
\end{grammar}
\begin{enumerate}
	\item Expression statements are either empty or consist of an expression.
	\item These effects of one statement are always completed before the next is executed.
	\item This guarantee is not valid in cases where explicit parallelization is used.
	\item Empty expression statements are used for loops and if statements where not action is to be taken.
\end{enumerate}


\subsection{Statement Block}
\begin{grammar}
<block> ::= \{ <compound_statement> \}
	\alt \{ <block> <compound_statement>\}

<compound_statement> ::= <declaration>
	\alt <statement>
	\alt <compound_statement> ; <declaration>
	\alt <compound_statement> ; <statement>
\end{grammar}
\begin{enumerate}
	\item A statement block is a collection of statements declarations and statements.
	\item If the declarations redefine any variables that were already defined outside the block, the new definition of the variable is considered for the execution of the statements in the block.
	\item Outside the block, the old definition of the variable is restored.
\end{enumerate}

\subsubsection{Branch Statements}
\begin{grammar}
	<branch_statement> ::= if ( <expression> ) <statement> fi
		\alt if ( <expression> ) <statement> else <statement> fi
\end{grammar}
\begin{enumerate}
	\item Branch statement are used to select one of several statement blocks based on the value of an expression.
\end{enumerate}

\subsection{Loop Statements}
\begin{grammar}
<loop_statement> ::= while ( <expression> ) <statement>
	\alt for ( <identifier> = <expression> to <expression> ) <statement>
	\alt for ( <assignment_expression> = <expression> to <expression> by <expression> ) <statement>
	\alt for ( <expression>; <expression>; expression> ) <statement>
\end{grammar}
\begin{enumerate}
	\item Loop statements specify the constructs used for iteration and repetition.
\end{enumerate}

\subsection{Parallel Statements}
\begin{grammar}
<parallel_statement> ::= parallel <block>
\end{grammar}
\begin{enumerate}
	\item The parallel block is for setting up concurrent execution constructs.
\end{enumerate}

\subsection{Jump Statements}
\begin{grammar}
<jump_statement> ::= break <integer-literal>\lit*{$_{opt}$}
	\alt continue
\end{grammar}
\begin{enumerate}
	\item Jump statements are used to break out of a loop or to skip the current iteration of a loop.
\end{enumerate}

\subsection{Return Statements}
\begin{grammar}
	<return_statement> ::= return 
	\alt return <expression>
\end{grammar}
\begin{enumerate}
	\item Return statements are used to denote the end of function logic and the also to specify the value to be returned by a call to the function in question.
\end{enumerate}
