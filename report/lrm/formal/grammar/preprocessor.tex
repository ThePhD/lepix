\section{Preprocessor}
\begin{grammar}
	<preprocessor_directive> ::= \#define <identifier> <expression>
	\alt \#ifdef <identifier>
	\alt \#ifndef <identifier>
	\alt \#endif
	\alt \#import <identifier>
	\alt \#import ‘‘<file_name>’’
	\alt \#import string <file_name>
\end{grammar}
\begin{enumerate}
	\item Before the source for a LePix program is compiled, the program is consumed by a preprocessor, which expands macro definitions and links libraries and other user-defines to the current file, as specified by appropriate preprocessor directives.
	\item \lstinline|define| macros create an alias for a value or expression.
	\item \lstinline|ifdef| and \lstinline|ifndef| macros are used to check if a particular alias has already been assigned. Import directives are used to link files/libraries with the current program.
\end{enumerate}
