\chapter{Tutorial}

\section{Invoking the Compiler}
Obtaining the \lepix{} compiler -- \lstinline|lepixc| -- in order to build \lepix{} programs with it is simple, once the dependencies are set up. It requires a working OCaml compiler of version \lstinline|>= 4.0.3|. Building requires \lstinline|ocamlbuild|, \lstinline|ocamlfind|, \lstinline|menhir| and \lstinline|OPAM| for an easy time, but if you are brave and willing to figure out the nightmare it takes to get these dependencies working on Windows than it can work on Windows machines as well.

From the root of the repository, run \lstinline|make -C source| to create the compiler, or \lstinline|cd| into the \emph{source} directory and invoke \lstinline|make| from there.

When the compiler is made, it will be within the \lstinline|./source| folder of the repository. An invocation without any arguments or filenames will explain to the user how to use it, like so:

\lstinputlisting{history/lepixc.txt}

Users can stack one-word options together using syntax like \lstinline|source/lepixc -ls|, which will pretty-print the Semantic Analysis tree and also output the LLVM IR code.

\section{Writing some Code}
Function definitions are fairly C-Like, with the exception that all type-annotations appear on the right-hand-side. Users can define variables and functions by using the \lstinline|var| and \lstinline|fun| keywords, respectively.

To receive access to the standard library, put \lstinline|import lib| in the program as well. It will end up looking somewhat like this:

\begin{lstlisting}
import lib

fun main () : int {
	lib.print_n("hello world");
	return 0;
}
\end{lstlisting}

One function must always be present in your code, and that is the \lstinline|main| function. It must return an \lstinline|int|. If the user does not return an integer value from main, then a \lstinline|return 0| will be automatically done for you. Also of importance is that functions can have their return type figured out from their return statements. For example, you can remove the \lstinline|: int| above and the code will still compile and run:

\begin{lstlisting}
import lib

fun main () {
	lib.print_n("hello world");
	return 0;
}
\end{lstlisting}

This goes for more than just the main function, but any function you define!
