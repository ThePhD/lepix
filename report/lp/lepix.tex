\documentclass[11pt,letterpaper]{report}
\usepackage[utf8x]{inputenc}
\usepackage{ucs}
\usepackage{amsmath}
\usepackage{amsfonts}
\usepackage{amssymb}
\usepackage{graphicx}
\usepackage{booktabs}
\usepackage{changepage}
\usepackage{hyperref}
\usepackage{listings}
\usepackage{color}

\definecolor{mygreen}{rgb}{0,0.6,0}
\definecolor{mygray}{rgb}{0.5,0.5,0.5}
\definecolor{mymauve}{rgb}{0.58,0,0.82}

\lstset{ %
	backgroundcolor=\color{white},   % choose the background color; you must add \usepackage{color} or \usepackage{xcolor}
	basicstyle=\footnotesize,        % the size of the fonts that are used for the code
	breakatwhitespace=false,         % sets if automatic breaks should only happen at whitespace
	breaklines=true,                 % sets automatic line breaking
	captionpos=b,                    % sets the caption-position to bottom
	commentstyle=\color{mygreen},    % comment style
	deletekeywords={...},            % if you want to delete keywords from the given language
	escapeinside={\%*}{*)},          % if you want to add LaTeX within your code
	extendedchars=true,              % lets you use non-ASCII characters; for 8-bits encodings only, does not work with UTF-8
	%frame=single,	                  % adds a frame around the code
	keepspaces=true,                 % keeps spaces in text, useful for keeping indentation of code (possibly needs columns=flexible)
	keywordstyle=\color{blue},       % keyword style
	language=C++,                 % the language of the code
	otherkeywords={*,...},           % if you want to add more keywords to the set
	numbers=left,                    % where to put the line-numbers; possible values are (none, left, right)
	numbersep=5pt,                   % how far the line-numbers are from the code
	numberstyle=\tiny\color{mygray}, % the style that is used for the line-numbers
	rulecolor=\color{black},         % if not set, the frame-color may be changed on line-breaks within not-black text (e.g. comments (green here))
	showspaces=false,                % show spaces everywhere adding particular underscores; it overrides 'showstringspaces'
	showstringspaces=false,          % underline spaces within strings only
	showtabs=false,                  % show tabs within strings adding particular underscores
	stepnumber=1,                    % the step between two line-numbers. If it's 1, each line will be numbered
	stringstyle=\color{mymauve},     % string literal style
	tabsize=5,	                  % sets default tabsize to 2 spaces
	title=\lstname,                  % show the filename of files included with \lstinputlisting; also try caption instead of title
	morekeywords={lib, super, function, uint, pixel, image, vec, mat, matrix, vector, unit, elseif}
}

\author{
	Fatima Koly (fak2116)\\
	Manager\\
	\texttt{fak2116@barnard.edu}
	\and
	Gabrielle Taylor (gat2118)\\
	Tester\\
	\texttt{gat2118@columbia.edu}
	\and
	Jackie Lin (jl4162)\\
	Tester\\
	\texttt{jl4162@columbia.edu}
	\and
	Akshaan Kakar (ak3808)\\ 
	Codegen \& Language Guru\\
	\texttt{ak3808@columbia.edu}
	\and
	ThePhD (jm3689)\\
	Codegen \& Language Guru\\
	\texttt{jm3689@columbia.edu}
	\\\\
	\url{https://github.com/ThePhD/lepix}
}
\title{LéPiX - Ceci n'est pas un Photoshop}
\begin{document}
	\renewcommand{\arraystretch}{1.1}
	\newcommand{\lepix}{LéPiX}
	\maketitle
	\tableofcontents
	\pagebreak
	\chapter{An Overview}
\section{Introduction}
Heterogeneous computing and graphics processing is an area of intense research. Many existing solutions -- such as C++ AMP \cite{clang-cxx-amp} and OpenCL \cite{opencl} -- leverage the power of an existing language and add preprocessors and software libraries to connect a user to allow code to be run on the GPU. Due to its massively parallelizable nature, code executed on the GPU can be orders of magnitude faster, but comes at the cost of having to master a specific programming library and often learn new framework-specific or platform-specific language subsets in order to compute on the GPU.

\section{Enter \lepix{}}
We envision \lepix to be a graphics processing language based loosely on a subset of the C language. Using an imperative style with strong static typing, we plan to support primitives that enable quick and concise programs for image creation and manipulation. The most novel feature we have planned for the language is the ability to compile to both the CPU as well as the GPU. The reason is to enable high performance and ease the pain most notably found with trying to write applications which leverage the power of the GPU at the cost of a steep learning curve for explicit non-CPU device computation APIs, e.g. OpenGL Compute Shaders, DirectCompute, OpenCL, CUDA, and others. The final goal is to enable the writing of computer vision and computer graphics algorithms in \lepix{} with relative ease compared to other languages.
	\chapter{The \lepix{} Language}


\section{Language}
The language itself is meant to follow loosely from imperative C, but it subject to change as we refine our desired set of primitives and base operations. What follows is a loose definition of the primitive types, operations we would like to implement to get a baseline for the language, and how would would like to put those together syntactically and grammatically. All of these definitions will be mostly informal.

The goal of \lepix{} is to provide a strongly and statically typed language by which to perform image manipulation easily. The hope is that powerful algorithms can be expressed in the language by providing a useful set of basic types, including the concept of a pixel and a natively slice-able matrix type that will serve as the basis for an image.

\begin{itemize}
	\item Primitive Data Type: the core types defined by the language itself and given a set of supportable operations
	\item Built-ins: some of the built-in functions to help users  
	\item Function Definitions: how to define a function in the language and use it
	\item Operators: which operators put into the language and operate on the primitive types
\end{itemize}

\subsection{User Defined Types}
Support for user-defined types is planned, but will be a stretch goal (\ref{sec:stretch user defined types}). We want to support having user-defined types that can be used everywhere, and for it to be able to overload the conceivable set of all operations.

\subsection{Syntax}
This is a basic guide to the syntax of the language. We are striving to develop a C-Like imperative language. It will follow many of the C conventions, but with some differences that we think will better fit the domain we are striving to work within. As we do not have a formal grammer just yet, we will present potential programs that we wish to allow to generate appropriate code. You can find these example programs in \ref{sec:examples}. Below are some quick points about the \lepix{} language.

\paragraph{Namespacing}
As a stretch goal, \lepix{} will attempt to support namespacing, to avoid record collision problems as present in OCaml (without the use of modules) and to formalize the good practice of of prepending the short name of the module / library to all functions in C code. Other languages have explicit support. \lepix{} will attempt to encourage code sharing and reuse by including the use of namespacing.

\paragraph{Keywords}
The following words will be reserved for use with the language: \lstinline|namespace, struct, class, typename, typedef, for, while, break, if, elseif, else, void, unit, int, float, uint, pixel, image, vec, vector, mat, matrix|. In addition, all identifiers containing \verb|__| (two underscores) are reserved for the use of the compiler and the standard library.

The standard library reserves the usage of the namespace \lstinline|lib|. The language will place intrinsics and built-ins within the \lstinline|lpx| namespace.

\paragraph{Function Definitions}
\label{par:function-definitions}
Typical function definitions will follow a usual C-style syntax. An example in pseudo-lexer code:
\lstinputlisting{code/func-def.lex}

Ideally, we would like the order of function definition not to matter, so long as it appears \emph{somewhere} in the whole source code listing. Early versions of the \lepix{} compiler might require definition before use, for sanity purposes.

As a stretch goal, there are plans for lambda functions (anonymous function values) to be generated by a much more terse syntax.

\paragraph{Control Flow}
Control flow will follow a C-style syntax as well. An example in pseudo-lexer code:
\lstinputlisting{code/flow-control.lex}

\paragraph{Operations}
The \lepix{} language will support most of the basic mathematical operators. Other operators will be provided VIA functions. These include:

\begin{itemize}
	\item Mathematical - \verb|plus (+), minus (-), multiply (*),|\\
	\verb|divide (/), modulus (%), power of (**)|
	\item Logical - \verb+and (&&), or (||), less than (<),+\\
	\verb+greater than (>), equal to (==), not equal to (!=)+\\
	\verb+negate (!)+
\end{itemize}

Some of these will use both the symbol and the name, such as "not" for negation. Support for bitwise operations, such as left shift and right shift as well as bitwise and / bitwise or, will come as a stretch goal, depending on whether we can handle these basics. Ternary conditionals may also prove useful, but will not be immediately supported.

\paragraph{Comments}
Single-line comments will begin with \lstinline|//|. Multi-line and \emph{nestable} comments will begin with \lstinline|/**/| will be supported as well.

\subsection{Built-ins}
Some useful built-in functions that will be provided with the language:

\begin{itemize}
	\item Trigonometric functions: \lstinline|lib.sin|, \lstinline|lib.cos|, \lstinline|lib.tan|, \lstinline|lib.asin|, \lstinline|lib.acos|, \lstinline|lib.atan|, \lstinline|lib.atan2|
	\item Power Functions: \lstinline|lib.sqrt|, \lstinline|lib.cbrt|, \lstinline|lib.pow|
	\item Exponential Functions: \lstinline|lib.expe|, \lstinline|lib.exp2|, \lstinline|lib.loge|, \lstinline|lib.log2|, \lstinline|lib.log10|
\end{itemize}

All degree arguments will come in radians. Other functions that operate on built-in types will be provided as library functions, to allow for replacement if necessary.

\subsection{Primitive Types}
Primitives are integral types and multi-dimensional array types. Vectors, Matrices, and Pixels are all subsets of N-dimensional array types. String will be presented as a built-in type, but may be implemented either as a built-in or just an always-included library type. The purpose of string will be specifically to handle reading in and writing out from the file system, as that is the only way to handle such a case. See \ref{tab:primitives} for details.

\begin{table}[htp]
	\begin{adjustwidth}{-2.15cm}{}
	\caption{Primitive Types in \lepix{}}
	\label{tab:primitives}
	\begin{tabular}{@{}lll@{}}
		\toprule
		Type                                                                   & variants                                                                                                                             & Purpose                                                                                                                                                                                            \\ \midrule
		void                                                                   & nothing                                                                                                                              & \begin{tabular}[c]{@{}l@{}}built in single-value /\\ empty type\end{tabular}                                                                                                                       \\
		int\#                                                                  & \begin{tabular}[c]{@{}l@{}}\# can be 8/16/32/64\\ (defaults to 32 without name)\end{tabular}                                         & signed integral type                                                                                                                                                                               \\
		uint\#                                                                 & \begin{tabular}[c]{@{}l@{}}\# can be 8/16/32/64\\ (defaults to 32 without name)\end{tabular}                                         & unsigned integral type                                                                                                                                                                             \\
		float\#                                                                & \begin{tabular}[c]{@{}l@{}}\# can be 16/32/64\\ (defaults to 32 without name)\end{tabular}                                           & floating type                                                                                                                                                                                      \\
		\textless{}type name\textgreater{[} \textless{}optional \#\textgreater {]} & \begin{tabular}[c]{@{}l@{}}array type; can be\\ multidimensional by \\ adding {[} \textless{}optional \# \textgreater {]}\end{tabular} & \begin{tabular}[c]{@{}l@{}}basic primitive that will \\ help us represent an \\ image in its decoded form;\\ can be sliced to \\ remove 1 dimension\end{tabular}                                   \\
		vec\#\textless{}type name\textgreater                                    & \begin{tabular}[c]{@{}l@{}}aliases for 1-dimensional\\ fixed-size arrays\end{tabular}                                                & base type for pixels                                                                                                                                                                               \\
		mat\#\textless{}type name\textgreater                                    & \begin{tabular}[c]{@{}l@{}}aliases to 2-dimensional\\ fixed-size arrays\end{tabular}                                                 & can be sliced into vectors                                                                                                                                                                         \\
		string                                                                 & \begin{tabular}[c]{@{}l@{}}utf8-encoded string; \\ potentially more later\end{tabular}                                               & \begin{tabular}[c]{@{}l@{}}primarily, to address \\ the file system;\\ not really interested in \\ complex text handling;\\ basically treated as an array \\ of specially-typed uint8\end{tabular} \\
		pixel                                                                  & \begin{tabular}[c]{@{}l@{}}rgba (red-green-blue-alpha);\\ hsv (hue saturation value)\end{tabular}                                    & \begin{tabular}[c]{@{}l@{}}in the future, this will be something\\ that will need to be customizable\\ to support varying image types \\ of different bit depths\end{tabular}                      \\ \bottomrule
	\end{tabular}
\end{adjustwidth}
\end{table}
	\chapter{Examples}
\label{sec:examples}

The following are some examples of programs that we would like to be written in \lepix{}. This does not reflect final syntax and is mostly based on equivalent or near-equivalent C-style code:

\begin{lstlisting}[caption={red.lpx},label=list:red]
function image red(int width, int height) {
	// Create an image of a specific width / height
	image ret = image(width, height);
	for(int i = 0; i < width;  i++) {
		for(int j = 0; j < height; j++) {
			// r, g, b creation
			ret[i][j] = pixel(1,0,0);
		}
	}

	return ret;
}

function int main () {
	image img = red(img);
	lib.save(img);
	return 0;
}
\end{lstlisting}


The red example shows up some simple conditionals in a for loop to write all-red to an image. It is not the most exciting code, but it has a lot of moving parts and will help us test several parts of the library, from how to call functions to basic iteration techniques.

\begin{lstlisting}[caption={flip.lpx},label=lst:flip]
function void flip(image img) {
	for (int ri = 0; ri < img.height / 2; ri++) {
		// matrix slicing: get back array from index
		pixel[] toprow = img[ri];
		// 0-based indexing
		pixel[] bottomrow = img[img.height - ri - 1];
		for (int ci = 0; ci < img.width; ++ci) {
			// generic swap call in library
			lib.swap(toprow[ci], bottomrow[ci]);
		}
	}
}

function int main () {
	image img = lib.read("meow.png");
	flip(img);
	lib.save(img);
	
	// implicit return 0:
	// we want this to be able to work with
	// command-line environments as well
}
\end{lstlisting}


This above code is a bit more complicated. It shows that we can save a slice of a matrix's (image's) row, operate on it, and even call the library function \lstinline|lib.swap| on its \lstinline|pixel| elements. It also demonstrates a string literal, and passing it to the \lstinline|lib.read| function to pull out a regular image from a PNG, and then saving that same image. It also shows off an implicit return 0 (we expect our programs to be run in the context of a shell environment, and to play nice with the existing C tools in that manner).

	\chapter{Codegen}

\section{LLVM IR}
The current goal for generating the code for this language is to use LLVM and serialize to LLVM IR. This will allow our language to work on a multiple of platforms that LLVM supports, provided we can successfully connect our AST / DAG with the our code generator.

\section{SPIR-V}
For the GPU, we still want to compile to LLVM IR. But, with the caveat that we make it work to push out to SPIR-V code using the Khronos LLVM <-> SPIR-V Bidirectional Translator\cite{spirv-llvm}\footnote{Available here: https://github.com/KhronosGroup/SPIRV-LLVM}. The good news is that everything from our source code processing steps to our DAG / AST generation can be done in OCaml. However, SPIR-V is new and OCaml bindings for this relatively new project are not something that is quite established: it is conceivable that our code generator will be written in C++ as opposed to any other language, simply because of the library power behind what is already present is written in C and C++, and interfacing that with OCaml might be exceptionally difficult. Granted, we could also write an OCaml Code Generator for SPIR-V from scratch, but this does not seem like a prudent use of our time.

It is also very important to note that the LLVM IR <-> SPIR-V Translator produces SPIR-V, which by itself cannot be run on anything. We would need a C or C++ compiled Vulkan Driver to take that bit of our program and run it in SPIR-V land. Furthermore, many operations -- such as data reading -- are not instructions we can exactly slot into SPIR-V code, and would require a bootstrapper of some sort nontheless.
	\chapter{Stretch Goals}
\section{User Defined Types}
\label{sec:stretch user defined types}
User-defined structures are a stretch goal of this project. The core idea is that if we can manage to create \lstinline|pixel|, \lstinline|vec#*|, \lstinline|mat#*|, etc. types using the language, we would be able to simply make this kind of functionality available to users. Currently, we plan to hard code these types in at the moment, however.

\section{Namespacing}
\label{sec:stretch namespacing}
Similar to user defined types, namespacing allows an element of organization to be brought to written code. Currently, we are going to hard code built ins to the \lstinline|lib| and \lstinline|lpx| namespaces.

\section{Named Parameters}
This is an entirely fluff goal to make it easier to call certain functions. The idea is that arguments not yet initialized by the ordered list of arguments to a function call can be specified out-of-order -- as long as others inbetween are defaulted -- by passing a \lstinline|name=(expression)| pairing, separated by commas like regular function arguments.

\section{Lambda Functions}
As mentioned in \ref{par:function-definitions}, we would like to support lambdas as a way of definition functions. Currently, we do not know what the most succinct and terse syntax for our language would be.

\section{Standard Library Implementation}
It would be nice to fill out a standard library implementation, to vet the \lepix{} compiler. Candidates would include some basic functions in the \lstinline|lib| namespace for manipulation of the \lstinline|image| type.

\section{Movies: Encoding}
If we can have built in types for \lstinline|image| and the like, then \lepix{} could theoretically handle movies by presenting to the user frames of data in sequential order. Doing this is orders of magnitude difficult.

\section{Windowing: Realtime Visuals}
Part of the magic of the graphics card is its ability to perform specific kinds of computation very quickly. It would be very beneficial to have some sort of way to display those visuals without having to serialize them to disk (e.g., a display function or a window of some sort which can be backed by a write-only image).

\section{Error Noises}
The compiler should make a snobby "Ouhh Hooo!" noise in french when the user puts in ill-formed code.

	\begin{thebibliography}{9}

\bibitem{spirv-llvm}
	Khronos Group,
	\emph{Khronos Registry: SPIR-V 1.1},
	April 18, 2016.
	https://www.khronos.org/registry/spir-v/

\bibitem{brook-gpu}
  Ian Buck, Tim Foley, Daniel Horn, Jeremy Sugerman, Pat Hanrahan, Mike Houston, Kayvon Fatahalian,
  \emph{Brook for GPUs},
  Stanford University,
  2016.
  https://graphics.stanford.edu/projects/brookgpu/arch.html

\bibitem{opencl}
	Khronos Group,
	\emph{OpenCL 2.2},
	April 12, 2016.
	https://www.khronos.org/opencl/

\bibitem{direct-compute}
	Chuck Walbourn, Microsoft,
	\emph{Compute Shaders},
	July 14, 2010.
	https://blogs.msdn.microsoft.com/chuckw/2010/07/14/directcompute/

\bibitem{clang-cxx-amp}
	Dillion Sharlet, Aaron Kunze, Stephen Junkins, Deepti Joshi,
	\emph{Shevlin Park: Implementing C++ AMP with Clang/LLVM and OpenCL},
	November 2012.
	http://llvm.org/devmtg/2012-11/Sharlet-ShevlinPark.pdf

\end{thebibliography}
\end{document}