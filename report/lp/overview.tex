\chapter{An Overview}
\section{Introduction}
Heterogeneous computing and graphics processing is an area of intense research. Many existing solutions -- such as C++ AMP \cite{clang-cxx-amp} and OpenCL \cite{opencl} -- leverage the power of an existing language and add preprocessors and software libraries to connect a user to allow code to be run on the GPU. Due to its massively parallelizable nature, code executed on the GPU can be orders of magnitude faster, but comes at the cost of having to master a specific programming library and often learn new framework-specific or platform-specific language subsets in order to compute on the GPU.

\section{Enter \lepix{}}
We envision \lepix to be a graphics processing language based loosely on a subset of the C language. Using an imperative style with strong static typing, we plan to support primitives that enable quick and concise programs for image creation and manipulation. The most novel feature we have planned for the language is the ability to compile to both the CPU as well as the GPU. The reason is to enable high performance and ease the pain most notably found with trying to write applications which leverage the power of the GPU at the cost of a steep learning curve for explicit non-CPU device computation APIs, e.g. OpenGL Compute Shaders, DirectCompute, OpenCL, CUDA, and others. The final goal is to enable the writing of computer vision and computer graphics algorithms in \lepix{} with relative ease compared to other languages.