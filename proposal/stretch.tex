\chapter{Stretch Goals}
\section{User Defined Types}
\label{sec:stretch user defined types}
User-defined structures are a stretch goal of this project. The core idea is that if we can manage to create \lstinline|pixel|, \lstinline|vec#*|, \lstinline|mat#*|, etc. types using the language, we would be able to simply make this kind of functionality available to users. Currently, we plan to hard code these types in at the moment, however.

\section{Namespacing}
\label{sec:stretch namespacing}
Similar to user defined types, namespacing allows an element of organization to be brought to written code. Currently, we are going to hard code built ins to the \lstinline|lib| and \lstinline|lpx| namespaces.

\section{Named Parameters}
This is an entirely fluff goal to make it easier to call certain functions. The idea is that arguments not yet initialized by the ordered list of arguments to a function call can be specified out-of-order -- as long as others inbetween are defaulted -- by passing a \lstinline|name=(expression)| pairing, separated by commas like regular function arguments.

\section{Lambda Functions}
As mentioned in \ref{par:function-definitions}, we would like to support lambdas as a way of definition functions. Currently, we do not know what the most succinct and terse syntax for our language would be.

\section{Standard Library Implementation}
It would be nice to fill out a standard library implementation, to vet the \lepix{} compiler. Candidates would include some basic functions in the \lstinline|lib| namespace for manipulation of the \lstinline|image| type.

\section{Movies: Encoding}
If we can have built in types for \lstinline|image| and the like, then \lepix{} could theoretically handle movies by presenting to the user frames of data in sequential order. Doing this is orders of magnitude difficult.

\section{Windowing: Realtime Visuals}
Part of the magic of the graphics card is its ability to perform specific kinds of computation very quickly. It would be very beneficial to have some sort of way to display those visuals without having to serialize them to disk (e.g., a display function or a window of some sort which can be backed by a write-only image).

\section{Error Noises}
The compiler should make a snobby "Ouhh Hooo!" noise in french when the user puts in ill-formed code.
