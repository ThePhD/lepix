\chapter{Examples}
\label{sec:examples}

The following are some examples of programs that we would like to be written in \lepix{}. This does not reflect final syntax and is mostly based on equivalent or near-equivalent C-style code:

\begin{lstlisting}[caption={red.lpx},label=list:red]
function image red(int width, int height) {
	// Create an image of a specific width / height
	image ret = image(width, height);
	for(int i = 0; i < width;  i++) {
		for(int j = 0; j < height; j++) {
			// r, g, b creation
			ret[i][j] = pixel(1,0,0);
		}
	}

	return ret;
}

function int main () {
	image img = red(img);
	lib.save(img);
	return 0;
}
\end{lstlisting}


The red example shows up some simple conditionals in a for loop to write all-red to an image. It is not the most exciting code, but it has a lot of moving parts and will help us test several parts of the library, from how to call functions to basic iteration techniques.

\begin{lstlisting}[caption={flip.lpx},label=lst:flip]
function void flip(image img) {
	for (int ri = 0; ri < img.height / 2; ri++) {
		// matrix slicing: get back array from index
		pixel[] toprow = img[ri];
		// 0-based indexing
		pixel[] bottomrow = img[img.height - ri - 1];
		for (int ci = 0; ci < img.width; ++ci) {
			// generic swap call in library
			lib.swap(toprow[ci], bottomrow[ci]);
		}
	}
}

function int main () {
	image img = lib.read("meow.png");
	flip(img);
	lib.save(img);
	
	// implicit return 0:
	// we want this to be able to work with
	// command-line environments as well
}
\end{lstlisting}


This above code is a bit more complicated. It shows that we can save a slice of a matrix's (image's) row, operate on it, and even call the library function \lstinline|lib.swap| on its \lstinline|pixel| elements. It also demonstrates a string literal, and passing it to the \lstinline|lib.read| function to pull out a regular image from a PNG, and then saving that same image. It also shows off an implicit return 0 (we expect our programs to be run in the context of a shell environment, and to play nice with the existing C tools in that manner).
